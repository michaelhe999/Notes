\documentclass[10pt,letterpaper,unboxed,cm]{article}
\usepackage{amsmath}
\usepackage{amssymb}
\usepackage{amsfonts}
\usepackage{graphicx}
\usepackage{listings}
\usepackage[margin=1in]{geometry}
\usepackage[normalem]{ulem}
\usepackage{latexsym}
\usepackage{xcolor}
\usepackage{color}
\usepackage{float}
\usepackage{hyperref}
\usepackage{tikz}
\usepackage{pgfplots}
\usepackage{pgfplotstable}
\usepackage{booktabs}
\usepackage{multirow}
\usepackage{siunitx}
\usepackage{enumitem}
\usepackage{subcaption}
\usepackage{caption}
\usepackage{cleveref}
\usepackage{mathtools}
\usepackage{bm}
\usepackage{algorithm}
\usepackage{algpseudocode}
\usepackage{algorithmicx}
\usepackage{tcolorbox}

\begin{document}
\begin{center}
    \textbf{\Large{CSE 110 Notes}}
\end{center}
\section{9/27}
Why is it so hard to properly get a software project done?\\
\begin{itemize}
    \item Scale - larger projects require more time; the longer it takes, the less likely the estimated time will be accurate
    \begin{itemize}
        \item More likely to be cancelled
    \end{itemize}
    \item Misunderstood and changing requirements - if software is already in operation, the cost to change it is higher
\end{itemize}
Class is meant to help deliver larger and better quality software projects\\
Quality control - early manufacturing revolved around:\\
\begin{itemize}
    \item Inspecting the product
    \item Fixing the product
    \item Reworking the production line
\end{itemize}
Led to Process-centric quality control\\
Still test the product, but also measure the process elements\\
Use cause-and-effect model to adjust production process\\
Statistical Process Control (SPC) - use statistics to track production variation\\
SE is Process-centric\\\\
What is a Software Process?\\
Produce quality software - what the customer wants, on time, under budget, no flaws\\
Steps include planning, execution, and measurement of product and process, and improvement\\\\
Discusses techniques for managing scale and risk/uncertainty\\
Process is just the beginning; also about quality decision-making\\
Needs good architecture, design, teamwork, and quality assurance\\\\
How to built what is needed, vs what is thought to be needed? Through frequent iteration and feedback from users\\
Robust code through good design and architecture\\\\
Project is self-decided\\
As long as choices make sense, you can get an A\\
Each student is graded on contributions to the team\\
Wisdom is better than quantity\\\\
Goals of the course: \\
Work effectively in a team using Agile development process\\
Design and document software systems according to stakeholder needs\\
Implement and debug complex software systems\\
Think about tradeoffs and risks\\\\
Courses need to teach technologies and principles, but principles need to be taught in context\\
Lecture focuses on principles, lab focuses on technologies\\
Team project: you choose the requirements\\
TA to manage project\\
Graded on ongoing quality and progress\\
Submit peer feedback every week, and TAs will give scores in Independence, Teamwork, and Technical contributions\\
\textbf{Focus groups} to provide insight regarding what users want\\
\textbf{Vision document} to say what you plan to create\\
A \textbf{mockup} will show the application in detail\\
In \textbf{five sprints}, build the app\\

\end{document}
